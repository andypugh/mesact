%% Generated by Sphinx.
\def\sphinxdocclass{report}
\documentclass[letterpaper,10pt,english,openany,oneside]{sphinxmanual}
\ifdefined\pdfpxdimen
   \let\sphinxpxdimen\pdfpxdimen\else\newdimen\sphinxpxdimen
\fi \sphinxpxdimen=.75bp\relax

\PassOptionsToPackage{warn}{textcomp}
\usepackage[utf8]{inputenc}
\ifdefined\DeclareUnicodeCharacter
% support both utf8 and utf8x syntaxes
  \ifdefined\DeclareUnicodeCharacterAsOptional
    \def\sphinxDUC#1{\DeclareUnicodeCharacter{"#1}}
  \else
    \let\sphinxDUC\DeclareUnicodeCharacter
  \fi
  \sphinxDUC{00A0}{\nobreakspace}
  \sphinxDUC{2500}{\sphinxunichar{2500}}
  \sphinxDUC{2502}{\sphinxunichar{2502}}
  \sphinxDUC{2514}{\sphinxunichar{2514}}
  \sphinxDUC{251C}{\sphinxunichar{251C}}
  \sphinxDUC{2572}{\textbackslash}
\fi
\usepackage{cmap}
\usepackage[T1]{fontenc}
\usepackage{amsmath,amssymb,amstext}
\usepackage{babel}



\usepackage{times}
\expandafter\ifx\csname T@LGR\endcsname\relax
\else
% LGR was declared as font encoding
  \substitutefont{LGR}{\rmdefault}{cmr}
  \substitutefont{LGR}{\sfdefault}{cmss}
  \substitutefont{LGR}{\ttdefault}{cmtt}
\fi
\expandafter\ifx\csname T@X2\endcsname\relax
  \expandafter\ifx\csname T@T2A\endcsname\relax
  \else
  % T2A was declared as font encoding
    \substitutefont{T2A}{\rmdefault}{cmr}
    \substitutefont{T2A}{\sfdefault}{cmss}
    \substitutefont{T2A}{\ttdefault}{cmtt}
  \fi
\else
% X2 was declared as font encoding
  \substitutefont{X2}{\rmdefault}{cmr}
  \substitutefont{X2}{\sfdefault}{cmss}
  \substitutefont{X2}{\ttdefault}{cmtt}
\fi


\usepackage[Bjarne]{fncychap}
\usepackage{sphinx}

\fvset{fontsize=\small}
\usepackage{geometry}


% Include hyperref last.
\usepackage{hyperref}
% Fix anchor placement for figures with captions.
\usepackage{hypcap}% it must be loaded after hyperref.
% Set up styles of URL: it should be placed after hyperref.
\urlstyle{same}

\addto\captionsenglish{\renewcommand{\contentsname}{Contents:}}

\usepackage{sphinxmessages}
\setcounter{tocdepth}{1}



\title{Mesa Configuration Tool}
\date{Jul 19, 2022}
\release{}
\author{John Thornton}
\newcommand{\sphinxlogo}{\vbox{}}
\renewcommand{\releasename}{}
\makeindex
\begin{document}

\pagestyle{empty}
\sphinxmaketitle
\pagestyle{plain}
\sphinxtableofcontents
\pagestyle{normal}
\phantomsection\label{\detokenize{index::doc}}


THIS REPO IS BEING DEPRECIATED DUE TO LACK OF USE BY THE LINUXCNC COMMUNITY.

The Mesa Configuration Tool creates LinuxCNC configuration files for
5i25, 6i25, 7i76e, 7i80db, 7i80hd, 7i92, 7i93, 7i95, 7i96, 7i96S, 7i97
and 7i98.


\chapter{Installing}
\label{\detokenize{install:installing}}\label{\detokenize{install::doc}}
Mesa Configuration Tool

\begin{sphinxadmonition}{note}{Note:}
Tested on Debian 10, and Linux Mint 20.2 but it should work on
other Debian type OS’s.
\end{sphinxadmonition}

\begin{sphinxadmonition}{note}{Note:}
Requires Python 3.6 or newer to work.
\end{sphinxadmonition}

Use the Debian deb for installing the Mesa Configuration Tool!!!

Download the \sphinxhref{https://github.com/jethornton/mesact/raw/master/mesact\_0.7.0\_amd64.deb}{deb}

Or use wget from a terminal

\begin{sphinxVerbatim}[commandchars=\\\{\}]
\PYG{n}{wget} \PYG{n}{https}\PYG{p}{:}\PYG{o}{/}\PYG{o}{/}\PYG{n}{github}\PYG{o}{.}\PYG{n}{com}\PYG{o}{/}\PYG{n}{jethornton}\PYG{o}{/}\PYG{n}{mesact}\PYG{o}{/}\PYG{n}{raw}\PYG{o}{/}\PYG{n}{master}\PYG{o}{/}\PYG{n}{mesact\PYGZus{}0}\PYG{o}{.}\PYG{l+m+mf}{7.0}\PYG{n}{\PYGZus{}amd64}\PYG{o}{.}\PYG{n}{deb}
\end{sphinxVerbatim}

If you get \sphinxtitleref{bash: wget: command not found} you can install it from a terminal with

\begin{sphinxVerbatim}[commandchars=\\\{\}]
\PYG{n}{sudo} \PYG{n}{apt} \PYG{n}{install} \PYG{n}{wget}
\end{sphinxVerbatim}

Check the readme.md file for the latest deb and md5sum.

Open the File Manager and right click on the file and open with Gdebi then install.

If you don’t have Gdebi installed you can install it from a terminal

\begin{sphinxVerbatim}[commandchars=\\\{\}]
\PYG{n}{sudo} \PYG{n}{apt} \PYG{n}{install} \PYG{n}{gdebi}
\end{sphinxVerbatim}

If the graphical version of gdebi has problems you can run it from a
terminal in the directory where you downloaded the deb with:

\begin{sphinxVerbatim}[commandchars=\\\{\}]
\PYG{n}{sudo} \PYG{n}{gdebi} \PYG{n}{mesact\PYGZus{}0}\PYG{o}{.}\PYG{l+m+mf}{6.1}\PYG{n}{\PYGZus{}amd64}\PYG{o}{.}\PYG{n}{deb}
\end{sphinxVerbatim}

If you don’t have LinuxCNC installed then the mesact Configuration tool
will show up in the Applications \textgreater{} Other menu otherwise it will be in
the CNC menu.

If you have problems try running from a terminal with:

\begin{sphinxVerbatim}[commandchars=\\\{\}]
\PYG{n}{mesact}
\end{sphinxVerbatim}

To flash firmware to the mesact you need to install
\sphinxhref{https://github.com/LinuxCNC/mesaflash}{mesaflash} from the LinuxCNC
repository.

To uninstall the mesact Configuration Tool right click on the .deb file
and open with Gdebi and select \sphinxtitleref{Remove Package}.

To upgrade the mesact Configuration Tool delete the .deb file and download
a fresh copy then right click on the .deb file and open with Gdebi and
select \sphinxtitleref{Reinstall Package}


\chapter{Basic Usage}
\label{\detokenize{basic:basic-usage}}\label{\detokenize{basic::doc}}
You can left click Check Config at any time to see if there are any errors.

Build Config will check for errors before build the configuration files.


\section{Machine Tab}
\label{\detokenize{basic:machine-tab}}\begin{enumerate}
\sphinxsetlistlabels{\arabic}{enumi}{enumii}{}{.}%
\item {} 
Enter a Configuration Name

\item {} 
Select Linear Units

\item {} 
Select Max Linear Velocity

\item {} 
Select the Mesa Board

\item {} 
Ethernet Boards you must select the IP Address 10.10.10.10 is recommened.

\item {} 
Boards like 5i25/6i25, 7i80, 7i92, 7i93, 7i98 to enable the Axes Tab
and the I/O Tab you need to select a firmware then select a daughter card.

\end{enumerate}


\section{Display Tab}
\label{\detokenize{basic:display-tab}}\begin{enumerate}
\sphinxsetlistlabels{\arabic}{enumi}{enumii}{}{.}%
\item {} 
Select a GUI

\item {} 
Select Position Offset

\item {} 
Select Position Feedback

\end{enumerate}


\section{Axes Tab}
\label{\detokenize{basic:axes-tab}}\begin{enumerate}
\sphinxsetlistlabels{\arabic}{enumi}{enumii}{}{.}%
\item {} 
Select Axis

\item {} 
Enter Scale, Minimum Limit, Maximum Limit, Maximum Velocity, Maximum
Acceleration

\item {} 
PID Settings select Default Values

\item {} 
Following Error select Default Values

\item {} 
For a Step and Direction select your drive or manually enter the Step
Time, Step Space, Direction Setup, Direction Hold times

\item {} 
For a Servo System select Default Values in Analog Output and enter
the Encoder Scale

\item {} 
Left Click Check Config to see if there are any errors

\end{enumerate}


\section{I/O Tab}
\label{\detokenize{basic:i-o-tab}}
The selected board will configure the Inputs and Outputs avaliable and
if input debounce is avaliable.

\#. Click Select for the I/O you want to use and select what you want it
to be used as.


\section{Spindle Tab}
\label{\detokenize{basic:spindle-tab}}
Used to configure an Analog PWM or Stepgen Spindle. For Digital Run, CW
and CCW type spindles use outputs.


\section{SS Cards Tab}
\label{\detokenize{basic:ss-cards-tab}}
If you have a Smart Serial Card attached you can configure it here.

\#. Select the Smart Serial Card and the page changes to that card where
you can make selections for that card


\section{GPIO Tab}
\label{\detokenize{basic:gpio-tab}}
Under Construction ATM, going to be where you could use the GPIO of a
pin directly. For example if you have an unused GPIO you could make it
either and input or output and use it.


\section{Tool Changer Tab}
\label{\detokenize{basic:tool-changer-tab}}
Yet to come


\section{Options Tab}
\label{\detokenize{basic:options-tab}}
Here you can select various options for your configuration and whether
to check for Mesaflash at startup or not.


\section{PLC Tab}
\label{\detokenize{basic:plc-tab}}
If your going to be using the Classicladder PLC you can set number of
items created for each type of bit.


\section{Pins Tab}
\label{\detokenize{basic:pins-tab}}
Displays the Terminal Block and pins for the selected card.

On most cards the Raw Output clicking Get Card Pinout will get a list of
pins.


\section{PC Tab}
\label{\detokenize{basic:pc-tab}}
You can get information about the PC CPU and NIC on the PC Info Tab.

If your using a Mesa Ethernet card you can test your NIC speed and get
the Packet Time and compare that to Threshold to see if your NIC and CPU
are fast enough at the current Servo Period.


\chapter{Machine Tab}
\label{\detokenize{machine:machine-tab}}\label{\detokenize{machine::doc}}
\noindent{\hspace*{\fill}\sphinxincludegraphics[scale=0.75]{{machine-tab-01}.png}\hspace*{\fill}}


\section{Menu}
\label{\detokenize{machine:menu}}

\subsection{File}
\label{\detokenize{machine:file}}\begin{itemize}
\item {} 
\sphinxtitleref{Open .ini File} \sphinxhyphen{} Opens a file selector so you can pick an ini file
to load, same as the Tool Bar button

\end{itemize}


\subsection{Tools}
\label{\detokenize{machine:tools}}\begin{itemize}
\item {} 
\sphinxtitleref{Check Config} \sphinxhyphen{} Checks the Configuration for errors

\item {} 
\sphinxtitleref{Build Config} \sphinxhyphen{} Builds the Congiguration after checking for errors

\end{itemize}


\subsection{Language}
\label{\detokenize{machine:language}}
Select the language to use, currently German is mostly translated.


\subsection{Help}
\label{\detokenize{machine:help}}\begin{itemize}
\item {} 
\sphinxtitleref{Tab Help} \sphinxhyphen{} Displays help information for the current tab, same as F1

\end{itemize}


\section{Tool Bar}
\label{\detokenize{machine:tool-bar}}\begin{itemize}
\item {} 
\sphinxtitleref{Open .ini File} \sphinxhyphen{} Opens a file selector so you can pick an ini file to load

\item {} 
\sphinxtitleref{Check Config} \sphinxhyphen{} Checks the Configuration for errors

\item {} 
\sphinxtitleref{Build Config} \sphinxhyphen{} Builds the Congiguration after checking for errors

\item {} 
\sphinxtitleref{Documents} \sphinxhyphen{} Opens the PDF Documents

\end{itemize}


\section{Machine Group}
\label{\detokenize{machine:machine-group}}\begin{itemize}
\item {} 
\sphinxtitleref{Configuration Name} \sphinxhyphen{} Any letter or number or underscore. Spaces are
replaced by an underscore.

\item {} 
\sphinxtitleref{File Path} \sphinxhyphen{} Displays the full path to the configuration.

\item {} 
\sphinxtitleref{Linear Units} \sphinxhyphen{} Select base units for the configuration.

\item {} 
\sphinxtitleref{Max Linear Velocity} \sphinxhyphen{} Set the Maximum Linear Velocity for all axes
combined in Linear Units per second.

\item {} 
\sphinxtitleref{Coordinates} \sphinxhyphen{} Displays the current configuration Coordinates by Axis

\end{itemize}


\section{Configuration Setup}
\label{\detokenize{machine:configuration-setup}}\begin{itemize}
\item {} 
\sphinxtitleref{Board Tab}

\item {} 
\sphinxtitleref{Board} \sphinxhyphen{} Select the main board being used.

\item {} 
\sphinxtitleref{IP Address} \sphinxhyphen{} If the main board is an Ethernet Board select the IP
address of the board.

\item {} 
\sphinxtitleref{Daughter Card} \sphinxhyphen{} After selecting the firmware you can select a daugher
card for which header you’re using. The header numbers are added to
the Daughter Card when a board is selected. At this time only one
daughter card is supported.

\item {} 
\sphinxtitleref{Options Tab}

\end{itemize}

After selecting a \sphinxtitleref{Firmware} the Options are populated. Select a lower
amount to free up GPIO on some boards.
\begin{itemize}
\item {} 
\sphinxtitleref{Step Generators}

\item {} 
\sphinxtitleref{PWM Generators}

\item {} 
\sphinxtitleref{Encoders}

\end{itemize}


\section{Firmware}
\label{\detokenize{machine:firmware}}
After selecting a board the Firmware combobox is populated with firmware
for that board.
\begin{itemize}
\item {} 
\sphinxtitleref{Read PD} \sphinxhyphen{} Read Pin Descriptions, gives more information than Read HMID

\item {} 
\sphinxtitleref{Read HMID} \sphinxhyphen{} Shows General Configuration Information

\item {} 
\sphinxtitleref{Flash} \sphinxhyphen{} After selecting a firmware this will flash the board

\item {} 
\sphinxtitleref{Reload} \sphinxhyphen{} After flashing firmware this will reload the new firmware

\item {} 
\sphinxtitleref{Verify} \sphinxhyphen{} After the board boots up this will verify the selected firmware

\item {} 
\sphinxtitleref{Copy} \sphinxhyphen{} Copies the contents of display window to the clipboard

\end{itemize}


\section{Backups}
\label{\detokenize{machine:backups}}\begin{itemize}
\item {} 
\sphinxtitleref{Enable Backups} \sphinxhyphen{} When this is checked a backup copy is saved to a
.zip file in the backups directory before building a new configuration.
The backup file is named with the date and time of the save.

\end{itemize}


\chapter{Info Tab}
\label{\detokenize{info:info-tab}}\label{\detokenize{info::doc}}
\noindent{\hspace*{\fill}\sphinxincludegraphics[scale=0.75]{{info-tab-01}.png}\hspace*{\fill}}

The Info Tab will have a diagram of the current board selected and may
have connection schematics and board notes. Also there are diagrams of
most daughter cards.

\noindent{\hspace*{\fill}\sphinxincludegraphics[scale=0.75]{{info-tab-02}.png}\hspace*{\fill}}

Mesa PDF manuals can be opened on the Manuals tab


\chapter{Display Tab}
\label{\detokenize{display:display-tab}}\label{\detokenize{display::doc}}
\noindent{\hspace*{\fill}\sphinxincludegraphics[scale=0.75]{{display-tab-01}.png}\hspace*{\fill}}


\section{Display Group}
\label{\detokenize{display:display-group}}\begin{itemize}
\item {} 
\sphinxtitleref{GUI} \sphinxhyphen{} Select the GUI you want to use

\item {} 
\sphinxtitleref{Position Offset} \sphinxhyphen{} Typically Relative is selected which includes any
offsets

\item {} 
\sphinxtitleref{Position Feedback} \sphinxhyphen{} Typically Commanded is selected, a servo system
when Actual is selected may bounce around and make the feedback change
a lot.

\item {} 
\sphinxtitleref{Maximum Feed Override} \sphinxhyphen{} Typically 1.2 is used

\end{itemize}


\section{G code Editor Group}
\label{\detokenize{display:g-code-editor-group}}\begin{itemize}
\item {} 
\sphinxtitleref{G code Editor} \sphinxhyphen{} Select from the installed editors

\end{itemize}


\section{Jog Slider Settings}
\label{\detokenize{display:jog-slider-settings}}\begin{itemize}
\item {} 
\sphinxtitleref{Minimum Linear Velocity} \sphinxhyphen{} The approximate lowest value the jog slider

\item {} 
\sphinxtitleref{Default Linear Velocity} \sphinxhyphen{} The default velocity for linear jogs, in ,
machine units per second

\item {} 
\sphinxtitleref{Maximum Linear Velocity} \sphinxhyphen{} The maximum velocity for linear jogs, in
machine units per second

\item {} 
\sphinxtitleref{Minimum Angular Velocity} \sphinxhyphen{} The approximate lowest value the angular
jog slider

\item {} 
\sphinxtitleref{Default Angular Velocity} \sphinxhyphen{} The default velocity for angular jogs, in
machine units per second

\item {} 
\sphinxtitleref{Maximum Angular Velocity} \sphinxhyphen{} The maximum velocity for angular jogs, in
machine units per second

\end{itemize}


\section{Axis Display Options}
\label{\detokenize{display:axis-display-options}}\begin{itemize}
\item {} 
\sphinxtitleref{Front Tool Lathe} \sphinxhyphen{} Normally a lathe is Front Tool that is when the
tool holder is on the users side of the spindle

\item {} 
\sphinxtitleref{Back Tool Lathe} \sphinxhyphen{} A Back Tool Lathe the tool holder is on the
opposite side of the spindle from the user side.

\end{itemize}


\chapter{Axes Tab}
\label{\detokenize{axes:axes-tab}}\label{\detokenize{axes::doc}}

\section{Axis Group}
\label{\detokenize{axes:axis-group}}\begin{enumerate}
\sphinxsetlistlabels{\arabic}{enumi}{enumii}{}{.}%
\item {} 
Select the type of Axis

\item {} 
Enter the Scale which is the number of pulses to move one user unit.
(user unit is either inch or mm)

\item {} 
Enter the Minimum Limit for the Axis (usually 0 for X or Y and the
amount of travel for the Z axis as a negative number

\item {} 
Enter the Maximum Limit for the Axis (usually max travel for X or Y
and 0 for Z)

\item {} 
Enter the Maximum Velocity in user units per second

\item {} 
Enter the Maximum Acceleration in user units per second per second

\item {} 
If the direction is backwards after testing check Reverse Direction

\end{enumerate}


\section{PID Settings Group}
\label{\detokenize{axes:pid-settings-group}}\begin{itemize}
\item {} 
Usually the Default Values are correct

\item {} 
If you change the Tread Period in the Options tab generate the PID
settings again.

\end{itemize}

The physical meaning of P=1/servo\_period (1000 for a 1 ms servo period)
is that any position errors are corrected before the next servo thread
invocation.

Anything greater than P=1/Servo\_period means you will over\sphinxhyphen{}correct.

Anything less than P=1/Servo\_period means you will under\sphinxhyphen{}correct.

Anything greater than P=2/Servo\_period means you will have oscillations.

If you are using PID feedback for a stepgen P=1/Servo\_period is pretty
much necessary. PID is still used with stepgens without encoders as it
has advantages over the built\sphinxhyphen{}in position mode

In addition you can use a bit of FF2 (FF2= seconds between position read
and new velocity write) usually about 0.0001 for Ethernet cards


\section{Following Error Group}
\label{\detokenize{axes:following-error-group}}\begin{enumerate}
\sphinxsetlistlabels{\arabic}{enumi}{enumii}{}{.}%
\item {} 
Usually the Default Values are correct

\end{enumerate}


\section{Homing Group}
\label{\detokenize{axes:homing-group}}
All entries are optional with the exception of a gantry configuration
with two or more axes with the same Axis Letter. In this case you must
enter the Home Sequence for all Joints used by the gantry.
\begin{enumerate}
\sphinxsetlistlabels{\arabic}{enumi}{enumii}{}{.}%
\item {} 
Home is usually 0

\item {} 
Home Offset can be used to move the joint off of a home switch

\item {} 
Home Search Velocity is the “fast” speed to find the home switch

\item {} 
Home Latch Velocity is the “slow” speed to get an accurate location
of the home switch

\item {} 
Home Final Velocity is the speed that joint moves to home positon,
if left blank the a rapid move is used

\item {} 
Home Sequence defines the order that the axes home, it must start
1 or 0 and is negative in the case of a gantry

\end{enumerate}

Step and Direction Drives

\noindent{\hspace*{\fill}\sphinxincludegraphics[scale=0.75]{{axes-tab-01}.png}\hspace*{\fill}}


\section{StepGen Settings Group}
\label{\detokenize{axes:stepgen-settings-group}}
Either enter in the values for your drive or select your drive from the
combo box. The Custom can be changed for your drive name if desired.

Analog Drives

\noindent{\hspace*{\fill}\sphinxincludegraphics[scale=0.75]{{axes-tab-02}.png}\hspace*{\fill}}


\section{Analog Output Group}
\label{\detokenize{axes:analog-output-group}}
Usually the Default Values are correct


\section{Encoder Group}
\label{\detokenize{axes:encoder-group}}
Enter the scale for your encoder for one user unit


\chapter{I/O Tab}
\label{\detokenize{io:i-o-tab}}\label{\detokenize{io::doc}}
\noindent{\hspace*{\fill}\sphinxincludegraphics[scale=0.75]{{io-tab-01}.png}\hspace*{\fill}}


\section{Inputs}
\label{\detokenize{io:inputs}}
Select the input function from the combo box. To deselect pick Select
from Not Used.

If you need to invert the sense of the input check Invert.

Some cards have a built in debounce function. If you check Debounce then
Invert is not avaliable and the same goes if you check Invert then
Debounce is not avaliable.

Inputs are enabled based on the board in the case of an all in one board
or the daughter card.


\section{Outputs}
\label{\detokenize{io:outputs}}
Select the output function from the combo box.  To deselect pick Select
from Not Used.



\renewcommand{\indexname}{Index}
\printindex
\end{document}